\documentclass{article}
\graphicspath{ {./images/} }

\usepackage{arxiv}
\usepackage{amsmath, bm, cite,graphicx,psfrag,pstricks,float,listings,color,subfig,booktabs, caption, xcolor, xparse, wrapfig}
\usepackage[utf8]{inputenc} % allow utf-8 input
\usepackage[T1]{fontenc}    % use 8-bit T1 fonts
\usepackage{hyperref}       % hyperlinks
\usepackage{cuted}
\usepackage{url}            % simple URL typesetting
\usepackage{booktabs}       % professional-quality tables
\usepackage{amsfonts}       % blackboard math symbols
\usepackage{nicefrac}       % compact symbols for 1/2, etc.
\usepackage{microtype}      % microtypography
\usepackage{lipsum}		% Can be removed after putting your text content\\
\setlength\columnsep{3em}

\title{A Bayesian Analysis on the accuracy of beam models}

%\date{September 9, 1985}	% Here you can change the date presented in the paper title
%\date{} 					% Or removing it

%\author{
%  David S.~Hippocampus\thanks{Use footnote for providing further
%    information about author (webpage, alternative
%    address)---\emph{not} for acknowledging funding agencies.} \\
%  Department of Computer Science\\
%  Cranberry-Lemon University\\
%  Pittsburgh, PA 15213 \\
%  \texttt{hippo@cs.cranberry-lemon.edu} \\
  %% examples of more authors
  %% \AND
  %% Coauthor \\
  %% Affiliation \\
  %% Address \\
  %% \texttt{email} \\
  %% \And
  %% Coauthor \\
  %% Affiliation \\
  %% Address \\
  %% \texttt{email} \\
  %% \And
  %% Coauthor \\
  %% Affiliation \\
  %% Address \\
  %% \texttt{email} \\
%}



\begin{document}


\maketitle

\begin{abstract}
	We use mathematical models to predict the behaviour of physical systems, whether it be structural beams, bridges, cells or circuits. This short paper aims to use how a Bayesian approach to assess quantitatively how well two different beam models, namely Euler-Bernolli beam theory and Finite Element Analysis (FEA), predict the deflection of a thin shelled structure.
\end{abstract}

% keywords can be removed
%\keywords{First keyword \and Second keyword \and More}

\section{Introduction}
Introduction

\section{Bayesian Analysis}
\label{sec:headings}

\section{Application to the Mechanics of Thin Shell Structures}

In this section we demonstrate the application of the proposed methodology to the deflection of thin shell structures. Thin shells are curved solids with one dimension significantly smaller than the other two. They are prevalent in nature, e.g., as insect wings or biological membranes and in engineering, most prominently in aerospace and automotive. Carefully designed curved thin shells have a load carrying capacity which is usually significantly higher than comparable flat structures.\\

\subsection{Problem Description}
We consider the composite beam shown in Figure 1 consisting of a gyroid core and two face plates. The gyroid is a triply periodic minimal surface with zero mean curvature and has recently been extensively explored in additive manufacturing applications, see e.g. Hussein et al. (2013); Abueidda et al. (2017). As known, cellular solids like the gyroid core can have mechanical properties that are orders of magnitude different from their constituent materials (Fleck et al., 2010). The length of the beam is $0.243m$, its height, i.e. distance between the top and bottom plates, is $0.1$ and its width is $0.1$. The gyroid core is described by the algebraic function\\

\begin{equation}
sin(\lambda x) cos(\lambda y) - sin(\lambda y) cos(\lambda z) - sin(\lambda y) cos(\lambda x) = 0
\end{equation}
with $\lambda = 20\pi$. The core and the two plates are modelled as thin shells and have a thickness of $t = 0.003$.
The beam is clamped at its left end, and at its right end the bottom plate is simply supported at a
distance $0.025$ away from the boundary. The Young’s modulus and the Poisson’s ratio are $E = 2.30GPa$
and $\nu = 0.3$. The top plate is subjected to a uniform pressure $f(x) = 3$ acting in the negative z
direction.


\lipsum[4] See Section \ref{sec:headings}.

\subsection{Getting data}
The gyroid beam is a complex geometry that can only be manufactured via additive methods, such as 3D printing. The gyroid beam was 3D printed using the facilities in the Dyson Centre for Engineering Design and tested using the instron machines at the Fatigue Lab at Cambridge University Engineering Department.\\

\begin{figure}
	\centering
	\fbox{\rule[-.5cm]{4cm}{4cm} \rule[-.5cm]{4cm}{0cm}}
	\includegraphics{}
	\caption{Sample figure caption.}
	\label{fig:fig1}
\end{figure}






\begin{equation}
\xi _{ij}(t)=P(x_{t}=i,x_{t+1}=j|y,v,w;\theta)= {\frac {\alpha _{i}(t)a^{w_t}_{ij}\beta _{j}(t+1)b^{v_{t+1}}_{j}(y_{t+1})}{\sum _{i=1}^{N} \sum _{j=1}^{N} \alpha _{i}(t)a^{w_t}_{ij}\beta _{j}(t+1)b^{v_{t+1}}_{j}(y_{t+1})}}
\end{equation}

\subsubsection{Headings: third level}
\lipsum[6]

\paragraph{Paragraph}
\lipsum[7]

\section{Examples of citations, figures, tables, references}
\label{sec:others}
\lipsum[8] \cite{kour2014real,kour2014fast} and see \cite{hadash2018estimate}.

The documentation for \verb+natbib+ may be found at
\begin{center}
  \url{http://mirrors.ctan.org/macros/latex/contrib/natbib/natnotes.pdf}
\end{center}
Of note is the command \verb+\citet+, which produces citations
appropriate for use in inline text.  For example,
\begin{verbatim}
   \citet{hasselmo} investigated\dots
\end{verbatim}
produces
\begin{quote}
  Hasselmo, et al.\ (1995) investigated\dots
\end{quote}

\begin{center}
  \url{https://www.ctan.org/pkg/booktabs}
\end{center}


\subsection{Figures}
\lipsum[10] 
See Figure \ref{fig:fig1}. Here is how you add footnotes. \footnote{Sample of the first footnote.}
\lipsum[11] 

\begin{figure}
  \centering
  \fbox{\rule[-.5cm]{4cm}{4cm} \rule[-.5cm]{4cm}{0cm}}
  \caption{Sample figure caption.}
  \label{fig:fig1}
\end{figure}

\subsection{Tables}
\lipsum[12]
See awesome Table~\ref{tab:table}.

\begin{table}
 \caption{Sample table title}
  \centering
  \begin{tabular}{lll}
    \toprule
    \multicolumn{2}{c}{Part}                   \\
    \cmidrule(r){1-2}
    Name     & Description     & Size ($\mu$m) \\
    \midrule
    Dendrite & Input terminal  & $\sim$100     \\
    Axon     & Output terminal & $\sim$10      \\
    Soma     & Cell body       & up to $10^6$  \\
    \bottomrule
  \end{tabular}
  \label{tab:table}
\end{table}

\subsection{Lists}
\begin{itemize}
\item Lorem ipsum dolor sit amet
\item consectetur adipiscing elit. 
\item Aliquam dignissim blandit est, in dictum tortor gravida eget. In ac rutrum magna.
\end{itemize}


\bibliographystyle{unsrt}  
%\bibliography{references}  %%% Remove comment to use the external .bib file (using bibtex).
%%% and comment out the ``thebibliography'' section.


%%% Comment out this section when you \bibliography{references} is enabled.
\begin{thebibliography}{1}

\bibitem{kour2014real}
George Kour and Raid Saabne.
\newblock Real-time segmentation of on-line handwritten arabic script.
\newblock In {\em Frontiers in Handwriting Recognition (ICFHR), 2014 14th
  International Conference on}, pages 417--422. IEEE, 2014.

\bibitem{kour2014fast}
George Kour and Raid Saabne.
\newblock Fast classification of handwritten on-line arabic characters.
\newblock In {\em Soft Computing and Pattern Recognition (SoCPaR), 2014 6th
  International Conference of}, pages 312--318. IEEE, 2014.

\bibitem{hadash2018estimate}
Guy Hadash, Einat Kermany, Boaz Carmeli, Ofer Lavi, George Kour, and Alon
  Jacovi.
\newblock Estimate and replace: A novel approach to integrating deep neural
  networks with existing applications.
\newblock {\em arXiv preprint arXiv:1804.09028}, 2018.

\end{thebibliography}


\end{document}
